\documentclass[a4paper]{article}

\usepackage[utf8]{inputenc}
\usepackage[T1]{fontenc}
\usepackage[french]{babel}
\usepackage{fullpage}
\usepackage{hyperref}
\usepackage{amssymb}
\usepackage{upgreek}

\title{
    Création d'un horaire d'examens\\
    \small Projet d'informatique fondamentale - INFO-F-302
}
\author{
    Titouan \bsc{Christophe} et Florentin \bsc{Hennecker}\\
    ULB BA3 info 2014-2015
}
\date{15 mai 2015}



\begin{document}
\maketitle
\tableofcontents

\section{\'Enonc\'e}
\subsection{Définitions}

Voici:
\begin{itemize}
  \item $X = \{x_1, ..., x_{n_X}\}$, l'ensemble des examens à organiser;
  \item $S = \{s_1, ..., s_{n_S}\}$, l'ensemble des salles disponibles;
  \item $c : S \mapsto \mathbb{N}$, l'application qui à toute salle associe sa capacité
  \item $E = \{e_1, ..., e_{n_S}\}$, l'ensemble des étudiants;
  \item $a : E \mapsto 2^X$, l'application qui à tout étudiant associe les examens qu'il doit passer;
  \item $P = \{p_1, ..., p_{n_P}\}$, l'ensemble des professeurs;
  \item $b : P \mapsto 2^X$, l'application qui à tout professeur associe les examens qu'il doit organiser;
  \item $T = \{t_1, ..., t_{n_T}\}$, l'ensemble des tranches horaires;
  \item $\tau_{x,t}$ est vrai ssi l'examen $x$ se déroule à la période $t$ avec $x \in X, t \in T$
  \item $\sigma_{x,s}$ est vrai ssi l'examen $x$ se déroule dans la salle $s$ avec $x \in X, s \in S$
  \item $\omega_{s,x}$ est vrai ssi la salle $s$ peut accueillir l'examen $x$ (donc si la capacité de la salle est assez élevée), avec $s \in s, x \in X$
\end{itemize}

\subsection{Objectif}
On cherche à définir une application $\mu : X \mapsto S \times {t_1, ..., t_{n_T}} \forall x \in X$,
qui associe à chaque examen la salle et l'horaire auquel il a lieu.

\textit{Si $\mu(x) = (s,t)$, alors l'examen $x$ a lieu dans la salle $s$ à l'heure $t$.}

\section{Questions}
\subsection{Définir la notion de correction pour une fonction $\mu$}
La fonction $\mu$ doit respecter les contraintes suivantes:
\begin{enumerate}
  \item Un étudiant ne peut avoir qu'un seul examen par tranche horaire;
  \item Un professeur ne peut avoir qu'un seul examen par tranche horaire;
  \item Une salle ne peut accueillir qu'un seul examen à la fois;
  \item Le nombre d'étudiants présentant un examen ne peut excéder la capacité de la salle qui l'accueille;
  \item Chaque examen doit avoir lieu une fois
  \item Chaque examen doit être surveillé par un professeur
  \item Chaque examen doit être passé par au moins un étudiant
\end{enumerate}

\subsection{Formaliser ces contraintes en langage mathématique}
\begin{enumerate}
  \item $\forall x_1,x_2 \in a(e_i), 1 \le i \le n_E: \mu(x_1) \mapsto (s_1,t_1), \mu(x_2) \mapsto (s_2,t_2) x_1 \neq x_2 \rightarrow t_1 \neq t_2 $
  \item $\forall x_1,x_2 \in b(p_i), 1 \le i \le n_P: \mu(x_1) \mapsto (s_1,t_1), \mu(x_2) \mapsto (s_2,t_2); x_1 \neq x_2 \rightarrow t_1 \neq t_2 $
  \item $\forall x_1,x_2 \in X: \mu(x_1) \mapsto (s_1,t_1), \mu(x_2) \mapsto (s_2,t_2); (x_1 \neq x_2) \land (t_1 = t_2) \rightarrow s_1 \neq s_2 $
  \item $\forall x \in X, 1 \le i \le n_E, \mu(x) \mapsto (s,t) \rightarrow \#\{e_i\ |\ x \in a(e_i)\} \le c(s)$
\end{enumerate}

\subsection{Construire une formule $\Phi$ en FNC}
On commence par exprimer chacune des contraintes énoncées en FNC :

\begin{enumerate}
  \item $ \bigwedge\limits_{t \in T} \,\,\,
          \bigwedge\limits_{e \in E} \,\,\,
          \bigwedge\limits_{x_i, x_j \in a(e), x_i \neq x_j} \,
          \lnot \tau_{x_i,t} \lor \lnot \tau_{x_j,t}$
  \item $ \bigwedge\limits_{t \in T} \,\,\,
          \bigwedge\limits_{p \in P} \,\,\,
          \bigwedge\limits_{x_i, x_j \in b(p), x_i \neq x_j} \,
          \lnot \tau_{x_i,t} \lor \lnot \tau_{x_j,t}$
  \item $ \bigwedge\limits_{t \in T} \,\,\,
          \bigwedge\limits_{s \in S} \,\,\,
          \bigwedge\limits_{x_i, x_j \in X, x_i \neq x_j} \,
          \lnot \tau_{x_i,t} \lor \lnot \tau_{x_j,t} \lor \lnot \sigma_{x_i,t} \lor \lnot \sigma_{x_j,t}$
  \item $ \bigwedge\limits_{x \in X} \,\,\,
          \bigwedge\limits_{s \in S} \,\,\,
          \lnot \sigma_{x,s} \lor \omega_{s,x}$
\end{enumerate}

\subsection{Implémenter et tester sur les exemples, et proposer éventuellement d’autres exemples}

\end{document}